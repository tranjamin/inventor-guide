\chapter{Annotations}

Annotations are tools that are used to add dimensions and other notes to a 3D model. In general, this approach is known as Model-Based Definition, and allows you to convey all the necessary information about a model without having to generate 2D sketches. Once you add annotations, they can be found in the Model Browser under the relevant folders.

\section{Tolerances}

When doing 3D modelling work, it is almost always necessary to incorporate tolerances to account for machining imprecisions.

\section{General Annotations}

\subsection{General Dimensions}

Add geometric dimensions to the 3D model by selecting \appcommand{General Annotation \then Dimension}, in a similar manner to 2D sketches. After selecting the features to dimension, you can then customise the text (and add symbols) and modify the style type of the dimension. Dimensions can then also be dragged around after creation, and using \hotkey{Right Click} on an annotation 

\subsection{Holes and Threads}

Dimension holes and threads by selecting \appcommand{General Annotation \then Hole/Thread Note}. Pair this with the \appcommand{3D Model \then Hole} feature to ensure that all holes are properly documented.

\subsection{Surface Textures}

The \appcommand{General Annotation \then Surface Texture} feature creates annotations that describe how a surface is to be machined, textured and finished. The format of these annotations abides typically by the ISO standard.

\section{Exporting Annotations}
