\chapter{3D Modelling}

This is where all the magic happens! These tools transform 2D sketches into 3D features, surfaces and more. Is it integral for any 3D modeller to understand how to effectively utilise the many tools at their disposal. There will usually be many ways to model an object, but good practices dictate to do it in a way that is the most robust and flexible.

This section will outline the 3D modelling tools which are available in Inventor, as well as when they should be used and the meanings of the numerous options each one is equipped with. In addition, the best practices of modelling will be given.

\section{Creating Features}

\subsection{Types of Features: At a Glance}
\begin{itemize}
\item Extrude: takes a cross-section from a 2D sketch and extrudes it into a prism.
\item Revolve: takes a section from a 2D sketch and rotates it around a specified axis.
\item Sweep: takes a cross-section from a 2D sketch and pulls it along a 2D/3D path to create a pipe-like solid.
\item Loft: creates an object that transitions between multiple cross-sections.
\item Coil: takes a profile and transforms it into a helical, spiraled or threaded object around a specified axis.
\item Emboss: takes text or other geometry and engraves or embosses it onto an object.
\item Decal: wraps an image to a face or multiple faces.
\item Rib: adds ribbed or webbed support walls to an object.
\item Import: allows you to import other CAD formats into your Modelling.
\item Unwrap: unwraps a solid body, often of sheet metal, until it becomes flat.
\item Derive: allows you to import an Inventor model to form the base of the current part, which you can then build on top of and is adaptive.
\end{itemize}

\subsection{Examples of When to use Features}
\begin{itemize}
    \item Extrude: you want to create basic objects, prisms and boxes.
    \item Revolve: you have a solid of revolution that you would like to model, such as a sphere or torus.
    \item Sweep: you would like to design an intricate slide, so you take the cross-section of the slide and sweep it around your 3D path.
    \item Loft: you identify using stress analysis two major load areas in a solid body, one large area and one small area. You would like to add structure between these areas to reinforce it, so you use a loft to create a solid that transitions from the large area to small area.
    \item Coil: you would like to manually model a spring, threaded screw or spiral.
    \item Emboss: you would like to engrave your brand name and logo into your products.
    \item Decal: you have a vehicle or other product that would be manufactured with an image printed on it.
    \item Rib: you have a corner that you would like to reinforce by adding a diagonal brace.
    \item Import: you have a model in a different format but want to use it as in Inventor part.
    \item Unwrap: you have a sheet metal object and want to unwrap it to give you a clear idea of how to manufacture the part prior to applying bends.
    \item Derive: you have a 3D modelled product and want to personalise it by adding an engraving. Insert the generic part with the Derive tool, and then add an your engraving. Changes to the generic part will update in the new part, but adding an engraving in the new part will not add it to the old part.
\end{itemize}

\subsection{Common Options}
When using the Create tools above, most of them have options which allow you even more flexibility with 3D modelling:

\begin{itemize}

\item Surface Mode: Instead of creating a solid object, you can instead create a surface. For example, instead of a sweep creating a solid pipe it will create a hollow one.
\item Profiles: the cross-sections you want to be turned into 3D features. You can often choose multiple, and they don't always have to be in the same sketch, plane or even the same orientation.
\item Presets: if the Create tool has many different parameters, you can create presets to speed up your workflow
\item Direction: direction can either be default, flipped, symmetric (same distance in each direction) or asymmetric (different distance in each direction)
\item Solid body: a part can be comprised of several distinct bodies. When creating a feature, you can choose which solid body it will be incorporated into.
\item Output (Boolean): the type of operation being performed. You can join the 3D feature to an existing solid body, cut the feature away from the body, take the intersection between the feature and body or create an entirely new solid.
\item Numerical Values: when entering numerical values, you can often click the arrow beside the input box to quickly measure dimensions, access recently used values or tell Inventor to use an existing dimension as a reference (updating the dimension in the future will also update the feature).
\item Keep Sketch Visible: under the hamburger menu on the top right of the popup dialog, you can specify to keep the sketch visible after executing the feature. This allows it to be reused.

\end{itemize}

\subsection{Additional Options: Extrusions}

\begin{itemize}
    \item From (Start Plane): you can choose the plane the extrusion will start at - it does not have to be the sketch plane!
    \item Distance (End Plane): you can specify how far the profile extrudes by:
    
    \begin{itemize}
        \item Manual Distance: specify the numerical distance of extrusion
        \item Through All: the extrusion will pass through all objects (for cut or intersect extrusions only).
        \item To: select the plane or feature that the extrusion will end at. For curved planes, Alternate Solution specifies to choose the closest side of the curved surface.
        \item To Next: the extrusion will end at the next feature or plane. Said feature or plane should entirely overlay the cross-section profile.
    \end{itemize}

    \item Taper: the extrusion tapers slightly inwards or outwards

\end{itemize}

\subsection{Additional Options: Revolve}

\begin{itemize}
    \item From (Start Plane): you can choose the plane the revolution will start at - it does not have to be the sketch plane!
    \item Distance (End Plane): you can specify how far the profile revolves by:
    
    \begin{itemize}
        \item Manual Angle: specify the numerical angle of revolution.
        \item Full: a 360 degree revolution.
        \item To: select the plane or feature that the extrusion will end at. For curved planes, Minimum Solution specifies to choose the closest side.
        \item To Next: the extrusion will end at the next feature or plane. Said feature or plane should entirely overlay the cross-section profile.
    \end{itemize}

\end{itemize}

\subsection{Additional Options: Sweep}

\begin{itemize}
    \item Path: Choose the path (a 2D/3D curve or edge) for the profiles to follow. It does not necessarily have to intersect the profiles, but the path must intersect the profile plane.
    \item Behaviour: Determines how the profile follows the path:
    \begin{itemize}
        \item Follow Path: the profile is swept along the path, where every cross section taken normal to the path stays constant. Adjust the twist and taper as necessary - 360 degree twist corresponds to one single twist, and positive/negative taper corresponds to a widening/narrowing of the sweep away from the start point.
        \item Fixed: the profile is swept along the path, where the cross section taken parallel to profile plane stays constant. Twist and taper are not available.
        \item Guide: the profile is swept along a path in the same manner as Follow Path. A second guide rail is used to determine how the profile scales (either in one or both directions) along the path - it guides the outside of the sweep.
    \end{itemize}
    
\end{itemize}


\subsection{Additional Options: Loft}

\begin{itemize}
    \item Sections: the cross-sections for the loft to transition through. You need at least two profiles, however they do not have to be in parallel planes. You can also loft between two profiles on the same plane, which will create a flat (2D) feature. You can also loft to single points rather than sections by simply selecting a point.
    \item Rails: the paths for the loft to follow
    \item Behaviour: how the rails are treated - either as outer rails or center lines.
    \item Conditions: allows you to control what angle the loft meets the sections at (i.e tangent to the plane)
    \item Closed Loops: joins the last section to the first section to form a closed loop object.
    \item Merge Tangent Faces: blends together faces that end up tangent with each other.
    \item Transition: 

\end{itemize}

\subsection{Additional Options: Coil}

\begin{itemize}
\item Axis: the axis the sketch will coil about. An internal axis will create a threaded object, and an external axis will create a helical object.
\item Behaviour: controls the physical properties of the coil, such as pitch height and revolution. Allows you to create a spiral (no vertical displacement)
\end{itemize}

\subsection{Additional Options: Emboss}

\begin{itemize}
\item Behaviour: determines whether the emboss raises up or recesses into the object. There is a third option, Emboss/Engrave From Plane, which allows you to create an emboss/engraving where the depth changes. In this option, you can also transition from an emboss to an engraving, which is a very powerful tool.
\item Taper: allows you to determine how the profile tapers (scales) when Emboss/Engrave From Plane is enabled.
\item Appearance: allows you to automatically change the visual appearance of the emboss
\item Wrap to Face: if embossing on a curved face, allows the emboss to follow the curve around. With this enabled, the emboss will always have the same depth at every location.
\end{itemize}

\subsection{Additional Options: Decal}

\begin{itemize}
    \item Automatic Face Chain: the image automatically wraps over adjacent faces.
\end{itemize}

\subsection{Additional Options: Derive}

\begin{itemize}
\item Derive Style: determines how the solid bodies of the derived part import.
\item Status: allows you to select which elements of the part will be imported.
\item Options: more options which, most importantly, allows you to scale and flip the part before importing.
\end{itemize}

\subsection{Additional Options: Import}

\begin{itemize}
\item Import Options: some file types have additional options (beside the "Open" button) which allows you to control the unit imported, change the subformat and more.
\end{itemize}

\subsection{Additional Options: Rib}

\begin{itemize}
\item Style: allows you to specify whether the ribs are created parallel or normal to the sketch plane. This is not important for basic ribs, but further customisation usually requires the style to be Normal to Sketch Plane
\item Thickness: determines the thickness of the ribs. The direction this thickness refers to depends on the style.
\item Hold Thickness (Normal to Sketch Plane > Draft): specifies whether the quoted thickness is taken from the top or bottom of the taper
\item Draft Angle (Normal to Sketch Plane > Draft): adds a taper to the normal face of the rib.
\item Boss: allows you to create ribs for bosses (which are like standoffs and usually have ribs on both sides).
\item Distance: determines the thickness of the ribs in the other direction. 
\end{itemize}

\subsection{Additional Options: Unwrap}

\begin{itemize}
\item Auto Face Chain
\item Alignment
\item Behaviour
\end{itemize}

\section{Primitives}
Primitives provide shortcuts to easily create basic features: a box, sphere, cylinder or torus. These features just reuse the Create features detailed above.

\section{Modifying 3D Features}

You can apply the following modifiers to change already existing 3D features. It is often better practice to use one of these tools, such as the Hole feature, instead of incorporating a hole into your sketch for the 3D features.

\subsection{Modifiers: At a Glance}

\begin{itemize}
\item Hole: allows you to create holes of all kinds
\item Fillet: rounds out sharp edges and corners
\item Chamfer: applies a bevel to sharp edges and corners
\item Shell: hollows out an object
\item Draft: angles existing faces
\item Thread: adds a thread to faces (including holes)
\item Combine: takes two solid bodies and joins, cuts or intersects them (the same options as 3D Create Features)
\item Thicken/Offset: thickens or thins faces 
\item Split: Splits solids and faces apart along planes, sketches or surfaces
\item Mark: adds text and other markings to faces. In essence, this splits the face and uses the border lines to show the marking
\item Delete Face: removes selected faces, leaving the other faces on the 3D feature intact
\item Move Bodies: allows you to move around solid bodies in relation to each other
\item Bend Part: takes an object and bends it over a line
\item Copy Object: duplicates objects
\item Direct Edit: allows you to edit imported models
\end{itemize}

\subsection{Hole}

Holes can be created at points that have been drawn on 2D/3D sketches. Alternatively, by selecting the Allow Center Point Creation option under Input Geometry, points can be added freehand. They have the following options:

Hole Type: Holes can be regular, tapped (threaded), tapered or a clearance hole, depending on the utility required. To design to fit a certain type of fastener, screw standards and sizes can specified under the Fastener tab.

Seat Type: The hole can have a countersunk, counterbore/spotface or no seat.

Behaviour: Determines the direction of hole, when or at what distance the hole terminates, and the geometry of any tapers or seats. Under Advanced Options Extend Start, the hole can also be extended so it extrudes both directions.

\subsection{Fillet}

Selection Priority: 

Fillet Body:

Fillet Radius: 

Edge Sets:

Edge Rolling:

Face Chaining:


\section{Patterning 3D Features}
\cbcolor{ForestGreen}
In the same manner as 2D sketches, most 3D features can be patterned to streamline work. Mirror features about a plane, or repeat it in a rectangular or circular pattern. Patterns are associative, so changes made to one object reflect to the others. Follows the same principles as 2D patterns, however there are a few more additional options which are available:



\subsection{Pattern Orientation}
\cbcolor{BurntOrange}
Choose whether the pattern copies stay in a fixed orientation or become rotated. For a circular pattern, this dictates whether all object face the same cartesian direction or all face inwards. For a rectangular pattern, determines whether all objects face the same cartesian direction or all lie tangent to either the first or second direction (this is under the additional options).


\subsection{Path-Based Patterns}
Unlike 2D sketches, rectangular patterns can also be generated along paths, including curved paths. Simply choose a custom 2D/3D sketch (or the edge of a 3D feature) and select it as the direction of patterning.


\subsection{Sketch-Driven Patterns}
Inventor also allows you to place down duplicate features according to a sketch of points. The feature is duplicated for each point, with the base point of the feature being the point that maps on top of each sketch point. The reference faces dictate the orientation (the face for the duplicates to lie tangent to). These can both be adjusted.



\subsection{Creation Method}
\cbcolor{Maroon}
This defines the method Inventor uses to generate duplicates, and although this is usually unimportant certain situations require certain creation methods.

Optimised: This is the fastest method, so useful for large quantities of duplication. However, it does not permit overlapping duplicates or patterns that intersect with other faces. Under the hood, it duplicates the faces of the features.

Identical: The second fastest method, for when optimised is not possible. Under the hood, it duplicates the entire feature.

Adjust: The slowest method, with each duplication have potential adjustments made to it. Used for the preservation of design intent and constraints - where it is not needed to keep each instance exactly the same. The best use-case for this is for a part which has been extruded to "terminate at face". Each duplicate will also be set to the same termination condition, even if it results in objects being lengthened and shortened from the reference feature.


\mediumdifficulty
\section{Freeform 3D Modelling}

\ribbon{freeform_ribbon}

Freeform modelling is an alternative technique to 3D modelling, which lets us directly manipulate objects like they are plasticine. This style of modelling is more similar to the mesh-based design you may use in programs such as Blender, and although it can take practice to get used to, it is a much faster way of creating complex 3D shapes and surfaces. In freeforms, you deal with meshes - surfaces divided into many grids. This surface may wrap back into itself to form a 3D object, but freeforms will always be hollow; there is no mass to it.

This section will cover how to create freeforms and the various tools used to manipulate them. It will also give you tips on when to best use freeforms and the design process associated with them.

\example{freeform_example}

\subsection{Creating Freeform Models}
\easydifficulty

We can create freeforms from the \appcommand{3D Model \then Create Freeform} panel. This will then put us in the Freeform environment, allowing us to create additional freeforms and edit them. Typically, we start our freeform model with one 6 primitve shapes, depending on which one best resembles the final shape we want our freeform to take. These base shapes are as follows: box, plane, cylinder, sphere, torus \& quadball. Note that for spherical or revolved shapes, we usually prefer quadball to sphere because it deforms uniformly in every direction. 

If we want a more complex shape, we can also start a freeform from a face (a polygon-shaped plane) or base it off an existing 3D model or surface. For more complex freeforms that need to be integrated with regular 3D modelling, this is usually the way to go, because you can exactly define how the freeform will look initially. 

When creating a freeform, we can customise it in several ways. For primitive shapes, we can define the size of each dimension, or use the visual arrows to manipulate the shape as we see fit. We can also define if we want to enforce symmetric constraints in any direction (see \hyperref[subsection:freeform-symmetry]{Symmetry \& Mirroring}). Finally, we can specify the number of faces our freeform is split up into in each direction. This controls how fine our mesh is, and therefore how much detail we can add.

Once we've created our freeform, the Freeform ribbon will open. We can additional freeform objects in the one instance under \appcommand{Freeform \then Create Freeform}. Usually we add planes to cover up holes in our model, or add additional bodies that we will later join to form one body. Never use the same freeform instance to create multiple parts; you should create each part separately.

\subsection{Visual Tools}

There are four visual options available specifically for freeform modelling.

\appcommand{Freeform \then Check \then Add Distance}: \newline
Adds a constrained dimension between a feature of a freeform body and a plane. This is useful for Model-Based Definition paradigms. Note that this has to be done in Smooth Mode rather than Blocky Mode.

\appcommand{Freeform \then Tools \then Toggle Translucent}: \newline
Change the opacity of the model. 

\appcommand{Freeform \then Tools \then Select Through}: \newline
Enabling this option will allow you to highlight and select geometry that is behind other geometry. This option is initially disabled, meaning that only parts of the model you can see will be selectable. Toggle this if you want to select all geometry or only geometry visible from a certain plane.

\appcommand{Freeform \then Tools \then Toggle Smooth}: \newline
Toggle between the smooth freeform model and it's blocky representation. It is usually better practice to model in blocky mode, as it shows the underlying geometry and is easier to edit. Smooth mode is usually the final product, so is reserved for renders and the like. This is the "actual" shape of the freeform.

\example{freeform_smooth_mode}

\subsection{Editing Freeform Models}

After creating a freeform body, we can change its geometry with \appcommand{Freeform \then Edit \then Edit Form}. Select one more multiple bodies, faces, edges or points - it is convenient to drag select an area of the body and then using the filtering options to select only faces, only edges or only points. (Double clicking on an edge will instead select the edge loop it belongs to). These geometries can then be translated, rotated, and/or scaled in all 3 dimensions, morphing the surrounding geometry as needed. We can either specify these transformations in the dialog box or use the visual arrows and points. Take note of the undo, redo and reset buttons in the dialog box. When editing freeforms, also have options for Extrude Mode, Soft Modifications and Transformation Coordinates.

Make sure to be in blocky mode when editing models. If you're in smooth mode, it's near-impossible to be able to predict how changes affect the overall shape. The blocky representation makes it very clear what is being changed.

\subsubsection{The Nuances of the Extrude Mode}
\mediumdifficulty
\appoption{Extrude} is an option that instead of morphing the surrounding geometry will only change the highlighted geometry. This is most commonly used for faces. Typically in transformations, pulling a face up will also pull the surrounding faces up to maintain smoothness. In extrude mode this is not the case, and it will look like a regular extrusion. This creates sharp angles in the blocky model, which translates to concave curves. 

A useful feature of extrusions is to create shells. Extrusions can be cut into the existing geometry to create hollow sections, which can then lead to very interesting concave shapes. A common technique is to extrude a face in one direction and then immediately the other, which removes the solid mass but keeps the outer shell.

Extrusions also have some quirks when used with edges. In most cases, translating will not change its orientation (i.e there will be no rotation). However, after selecting the Extrude mode when making edge transformations, the resulting edge will associate with surrounding edges. Trying to transform this new edge will then pull along other edges in ways that were not previously possible. 

\example{freeform_extrude}

\subsubsection{The Coordinate System for Transformations}
\mediumdifficulty
An important note when editing freeform models is that the coordinate system used for transforming can be set to either the world coordinates, local coordinates or view coordinates. Most commonly the former two are used. We use world coordinates to move the geometry vetically or horizontally, while we use local coordinates to move orthogonally or tangent to the face - this is particulary useful for angled faces.

\subsubsection{Soft Modification}
Soft modifications allow for more gradual geometries, but as a trade off usually affect a wider amount of the freeform body. The type of modification (Circle, Rectangle, or Grow) determines the "range of influence" (i.e the area of the freeform body which is affected) and can be varied using parameters. For example, a \appoption{Circle} modification with radius 10mm will affect any geometry within 10mm of the entity being edited. Usually, larger influence radii correspond to more gradual change. The \appoption{Falloff} and \appoption{Gradient} determine how the change will be distributed through this area.

\subsubsection{Deleting Features}
\easydifficulty
We delete features under \appcommand{Freeform \then Edit \then Delete}. Typically, deleting edges will merge faces together, deleting faces will hollow out the freeform body and deleting points will merge all adjacent faces and edges.

\subsubsection{Aligning Freeforms}
We align points or symmetry planes to planes using \appcommand{Freeform \then Edit \then Align Form}. Note that for freeform models, planes are more difficult to define because we can't create planes based on freeform geometry. Therefore, it is often practical to align freeforms to existing planes. By doing this, we know where our freeform is in relation to the rest of the model, and can more easily create new planes (such as for symmetry operations).

\subsection{Modifying Freeforms}

This section covers the other tools used to modify freeforms.

\subsubsection{Inserting Edges and Points and Subdividing}

A common method used to modify freeforms is the introduction of additional geometry. This makes the the model more intricate and gives us greater control over how the blocky model translates to the smooth model. The three primary ways to introduce more detail is by inserting an edge, inserting points and subdividing faces.

In edge inserting (\appcommand{ Modify \then Insert Edge}), we split a face by offsetting one of the edges that bounds it. The location (from 0 to 1) represents how far down the face to insert the new edge. We can also choose if we want the new edge on one or both of the faces the reference edge lies on. 

If we want to make a new edge that is not parallel from an existing one, we can instead add two points and create a new edge between them (\appcommand{Modify \then Insert Point}) - we can also add a single point to bisect an existing edge. 

To split faces into many parts, we use the (\appcommand{Modify \then Subdivide}) tool. This allows us to break down down a face into equidistant rows and columns.

For each of these tools, we can either choose the simple or exact mode. \appoption{Simple} Mode will create the exact number of faces specified, however in doing so may change the overall shape. If you only care about getting the number of faces you want, use this mode. If you want to refine the detail of an existing model without yet changing the overall shape, use \appoption{Exact} Mode. This adds extra faces to surrounding geometry in order to preserve the original shape.

\subsubsection{Merging, Bridging and Creasing}
\mediumdifficulty

Two sets of open edges (those connected to only one edge) can be joined together using the \appcommand{Modify \then Merge Edge} feature. This acts in a similar way to the Loft feature in 3D modelling, creating a bridge between the two edges. The merging point can be defined as being at the edge set or in the middle of them. This is particularly useful for joining two holes together - remember we can double click to select the entire loop of an edge.

\appcommand{Modify \then Bridge} perform similarly, however they join two faces together without having to create a hole first. Between two bodies, this will create a solid bridge, however within a single body this will create a hole between the two faces. We can bridge from any number of faces to any other number of faces, and customise the bridge by specifying the number of faces it will have and the number of twists it will perform. This is also very similar to the Loft feature. Note that by default the preview for this feature is disabled because it is computationally expensive. This can be changed in the dialog box.

Freeforms convert blocky models to smooth models, removing any sharp edges. However, we may sometimes want these sharp edges for particular purposes. The \appcommand{Modify \then Crease Edges} tool allows us to specify which edges should not be smoothened. They can be uncreased in like with \appcommand{Modify \then Uncrease Edges}.

\subsubsection{Welding}
\easydifficulty

If there are two or more points on the freeform that we want to connect, we can do so by using \appcommand{Modifty \then Weld Vertices}. The weld can be made to be at the centre of the points or towards the second vertex. In addition, with multiple points the welds can be made to a specific tolerance. 

An edge weld is defined as a loop which cuts the freeform object in two, and is often the result of either basic modelling or features such as mirroring. To break apart a freeform, we use the \appcommand{Modify \then Unweld Edges} and then select the loop we want to split along. 


\subsubsection{Flattening and Thickening}

\appcommand{Modify \then Flatten} allows us to compress selected vertices into one plane. We can either define the plane we want the points to lie on, a corresponding parallel plane, or we can let Inventor compute a line of best fit automatically.

\appcommand{Modify \then Thicken} lets us add thickness to surfaces - in reality, this is not strictly true because the inside of our thicker surface is actually hollow. This tool takes a body and offsets it by a specified amount to add thickness (or remove existing thickness). This results in two copies of the same body, one slightly smaller inside the slightly larger one - this gives the thickness. We can choose not to join these together, or more typically join with either a sharp or a rounded edge. Typically this offset is applied in all directions (normal to the surface), but we can also choose a specific axis to add thickness to.

\subsubsection{Matching}
\mediumdifficulty

Edge Matching is a critical step in incorporating freeform models into CAD designs. Often, freeform models tend to be very quick to make, but difficult to do precisely - and are therefore difficult to design around. Usually what we do is design our freeform to what it approximately needs to be and then use this tool, \appcommand{Modifty \then Match Edges}, to line the freeform up accurately. It glues an edge or loop from a freeform to a 2D sketch or 3D geometry, ensuring these models can be integrated into larger projects. 

We can also set tolerances and flip the way to connect these edegs - for those familiar with topography, this creates a non-orientable surface homeomorphic to the mobius strip because we are introducing a twist. Therefore, we usually don't want to do this, and only use this setting when Inventor gets the default direction wrong. In addition, we usually avoid sharp corners because these are difficult for Inventor to model in 3D. 

A special case arises if the edge of the freeform is from a NURBS (Non-Uniform Rational B-Spline) Surface. These surfaces are usually imported from other softwares or created in the Surfaces menu and then transformed into freeform. If we are using these types of surfaces, we can define the continuity of the matching to be G0, G1 or G2.

Note that matching edges will create a "Matches" folder under the freeform object in the Part Navigator, and if any of the sketches change the edge matchings can be rematched or redefined here. It doesn't do this automatically because of how computationally expensive this process is. 

\example{freeform_matching}

\label{subsection:freeform-symmetry}
\subsection{Symmetry and Mirroring}
\easydifficulty

Like the corresponding tools for sketching and modelling, these tools allow us to define associative symmetries and mirrors in our freeform model. Under the \appcommand{Symmetry} subpanel, we can \appcommand{Symmetry \then Mirror} a body along a mirror plane. Any changes to one side will be reflected in its mirror. Touching geometry across the mirror plane is typically welded together. The same effect of associative features can be achieved through the \appcommand{Symmetry \then Symmetry} tool. This adds symmetry to two already existing features. We can remove symmetries if needed as well with \appcommand{Symmetry \then Clear Symmetry}.

A further option is to make a freeform body uniform under \appcommand{Check \then Make Uniform}. This removes the pinch points and attempts to smooth out the surfaces as much as possible. Unless there is a good reason not to, we usually run this command at the end of our modelling to make sure everything is uniform.

\subsection{How to Use Freeforms}

Freeform modelling can be really good for certain parts of the design proces, but lacks the necessary precision for many engineering applications. To use them effectively, we recommend following the below process:

\begin{enumerate}

\item
Identify where freeform modelling is useful. They are often used for complex 3D surfaces and shapes that don't necessarily need to be precise, or to join two components together. Freeform models also can't be dimensioned, so consider how a component is being manufactured.

\item
Create the surrounding model that you want to integrate your freeform into. This should include sketches and faces for it to connect to. Also define planes and axes to guide your freeform modelling.

\item
Choose the primitive that most closely aligns with that you want to design. Remember that for a spherical-based object that behaves the same in all directions, choose the quadball. Decide how detailed your model will be: the finer the mesh, the greater control you will have over the shape but the more computationally expensive it will be.

\item
Define any symmetries or mirrors in the design. These should be used whenever possible to maintain symmetry of the design if needed.

\item 
Edit and modify your model as necessary.

\item 
Ground your freeform model to the correct location relative to your model using the Align Form tool.

\item 
Use the Match Edge tool to connect the freeform to the rest of the model precisely.
\end{enumerate}

\subsection{Glossary of Freeform Terms}

Freeform object/model: a connected group of faces, edges and vertices/points.

Open surface: a surface which does not join in on itself to create volume. The boundary of this open surface is the edge that is only connected to a single face.

Loop: a series of edges connected by vertices which forms a closed chain (i.e forms the boundary of a surface)

Freeform geometry: the edges, points and faces that comprise a freeform model.

Mesh: the group of edges that make up the model.

\section{Surfaces}
\label{section: 3D Model Surface}

$ $

\section{Simplifying 3D Models}

$ $

\section{Plastic Parts}

$ $

\section{Insert}

$ $

\section{Harness}

$ $

\section{The Shape Generator}
Shape Generator is Autodesk Inventor's version of generative design, which is a powerful feature which takes a load scenario and creates highly optimised structural designs.

\section{Feature Management}

\section{Best Practices}