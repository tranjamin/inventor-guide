\section{3D Modelling}

This is where all the magic happens! Transform sketches into 3D features, surfaces and more.

\subsection{Creating Features}

$ $

\subsubsection{Types of Features}
\begin{itemize}
\item Extrude: takes a cross-section from a 2D sketch and extrudes it into a prism.
\item Revolve: takes a section from a 2D sketch and rotates it around a specified axis.
\item Sweep: takes a cross-section from a 2D sketch and pulls it along a 2D/3D path to create a pipe-like solid.
\item Loft: creates an object that transitions between multiple cross-sections.
\item Coil: takes a profile and transforms it into a helical, spiraled or threaded object around a specified axis.
\item Emboss: takes text or other geometry and engraves or embosses it onto an object.
\item Decal: wraps an image to a face or multiple faces.
\item Rib: adds ribbed or webbed support walls to an object.
\item Import: allows you to import other CAD formats into your Modelling.
\item Unwrap: unwraps a solid body, often of sheet metal, until it becomes flat.
\item Derive: allows you to import an Inventor model to form the base of the current part, which you can then build on top of and is adaptive.
\end{itemize}

\subsubsection{Examples of When to use Features}
\begin{itemize}
    \item Extrude: you want to create basic objects, prisms and boxes.
    \item Revolve: you have a solid of revolution that you would like to model, such as a sphere or torus.
    \item Sweep: you would like to design an intricate slide, so you take the cross-section of the slide and sweep it around your 3D path.
    \item Loft: you identify using stress analysis two major load areas in a solid body, one large area and one small area. You would like to add structure between these areas to reinforce it, so you use a loft to create a solid that transitions from the large area to small area.
    \item Coil: you would like to manually model a spring, threaded screw or spiral.
    \item Emboss: you would like to engrave your brand name and logo into your products.
    \item Decal: you have a vehicle or other product that would be manufactured with an image printed on it.
    \item Rib: you have a corner that you would like to reinforce by adding a diagonal brace.
    \item Import: you have a model in a different format but want to use it as in Inventor part.
    \item Unwrap: you have a sheet metal object and want to unwrap it to give you a clear idea of how to manufacture the part prior to applying bends.
    \item Derive: you have a 3D modelled product and want to personalise it by adding an engraving. Insert the generic part with the Derive tool, and then add an your engraving. Changes to the generic part will update in the new part, but adding an engraving in the new part will not add it to the old part.
\end{itemize}

\subsubsection{Common Options}
When using the Create tools above, most of them have options which allow you even more flexibility with 3D modelling:

\begin{itemize}

\item Surface Mode: Instead of creating a solid object, you can instead create a surface. For example, instead of a sweep creating a solid pipe it will create a hollow one.
\item Profiles: the cross-sections you want to be turned into 3D features. You can often choose multiple, and they don't always have to be in the same sketch, plane or even the same orientation.
\item Presets: if the Create tool has many different parameters, you can create presets to speed up your workflow
\item Direction: direction can either be default, flipped, symmetric (same distance in each direction) or asymmetric (different distance in each direction)
\item Solid body: a part can be comprised of several distinct bodies. When creating a feature, you can choose which solid body it will be incorporated into.
\item Output (Boolean): the type of operation being performed. You can join the 3D feature to an existing solid body, cut the feature away from the body, take the intersection between the feature and body or create an entirely new solid.
\item 

\end{itemize}

\subsubsection{Extrude}
Takes a 2D cross-section and extrudes it into a prism. 

Profiles:
These are the cross-sections which are to be extruded. Select them by clicking on any desired closed areas within a single sketch.

From:
Choose the plane that the extrusion will start at (it does not have to be the sketch plane, although it will default to this).

Direction:
The extrusion can be set as in either direction, symmetric in both directions or asymmetric in the two directions.

Distance:
The distance of extrusion can be set as:

\begin{itemize}
    \item Manual Distance: specify the numerical distance of extrusion (for asymmetric extrusions you will need to specify a distance for each direction). You can click the arrow beside the input box to quickly measure dimensions, use an existing dimension for reference or use recently used values.
    \item Through All: the extrusion will pass through all objects (also can be used for cut or intersect extrusions).
    \item To: select the plane or feature that the extrusion will end at. If the feature is has multiple possible termination points (such as the closest and furthest side of a cylinder), select Alternate Solution to end the extrusion at the closest face of the feature
    \item To Next: the extrusion will end at the next feature or plane. Said feature or plane should entirely overlay the cross-section profile.
\end{itemize}

Taper:
The extrusion tapers slightly inwards (negative values) or outwards (positive values).

Settings:
Additional settings can be accessed by the hamburger menu on the top right of the popup dialog. Of most note is the option to keep the sketch visible after extrusion - otherwise, the sketch will be consumed in the process.

\subsubsection{Revolve}
Revolves a section around a specified axis.

Profiles:
These are the cross-sections which are to be revolved. Select them by clicking on any desired closed areas within a single sketch.

From:
Choose the plane that the revolution will start at (it does not have to be the sketch plane, although it will default to this).

Direction:
The revolution can be set as in either direction, symmetric in both directions or asymmetric in the two directions.

Distance:
The distance of revolution can be set as:

\begin{itemize}
    \item Manual Angle: specify the numerical angle of revolution (for asymmetric extrusions you will need to specify a distance for each direction). You can click the arrow beside the input box to quickly measure dimensions, use an existing dimension for reference or use recently used values.
    \item Full: a 360 degree revolution
    \item To: select the plane or feature that the extrusion will end at. If the feature is has multiple possible termination points (such as the closest and furthest side of a cylinder), select Minimum Solution to end the revolution at the closest face of the feature.
    \item To Next: the extrusion will end at the next feature or plane. Said feature or plane should entirely overlay the cross-section profile.
\end{itemize}

\subsubsection{Sweep}
Takes a 2D cross section and sweeps it across a path to create a pipe-like feature.

Profiles:
These are the cross-sections which are to be sweep. Select them by clicking on any desired closed areas within a single sketch. They do not necessarily have to be connected but errors may occur if you try to sweep dispersed profiles along tricky paths (i.e if it causes the sweep to self-intersect)

Path:
Choose the path (a 2D/3D curve or edge) for the profiles to follow. It does not necessarily have to intersect the profiles, but the path must intersect the profile plane.

Behaviour:
Determines how the profile follows the path:

\begin{itemize}
    \item Follow Path: the profile is swept along the path, where every cross section taken normal to the path stays constant. Adjust the twist and taper as necessary - 360 degree twist corresponds to one single twist, and positive/negative taper corresponds to a widening/narrowing of the sweep away from the start point.
    \item Fixed: the profile is swept along the path, where the cross section taken parallel to profile plane stays constant. Twist and taper are not available.
    \item Guide: the profile is swept along a path in the same manner as Follow Path. A second guide rail is used to determine how the profile scales (either in one or both directions) along the path - it guides the outside of the sweep.
\end{itemize}

\subsubsection{Loft}
Creates an object that transitions between sketches.

Sections:
These are the cross-sections that the loft will transition through. You need at least two profiles, however they do not have to be in parallel planes. You can also loft between two profiles on the same plane, which will create a flat (2D) feature. You can also loft to single points rather than sections by simply selecting a point.


These are the cross-sections which are to be sweep. Select them by clicking on any desired closed areas within a single sketch. They do not necessarily have to be connected but errors may occur if you try to sweep dispersed profiles along tricky paths. The

Path:
Choose the path (a 2D/3D curve or edge) for the profiles to follow. It does not necessarily have to intersect the profiles, but the path must intersect the profile plane.

Behaviour:
Determines how the loft follows rails.

\begin{itemize}
    \item Rails: determines which points of the profile will connect to each other.
    \item Centerline: determines how the center of the loft moves.
    \item Area Loft: determines how the center of the loft moves, as well as the more control over intermediate cross sections.
\end{itemize}

Closed Loops and Tangent Faces:
You can select the Closed Loop checkbox to join the last section to the first, creating a closed loop object.

You can select the Tangent Faces checkbox to blend faces of the loft together which end up tangent with each other.

Conditions:
Allows you to control whether the loft meets each profile at a tangent or any other angle. If you have multiple sections in a loft, you can also specify the lofts to be smooth with each other. The weight of the sketch controls how quickly the loft moves to the specified angle.

Transition:
-------

\subsubsection{Coil}
Takes profile and rotates it around an axis to form a helical, threaded or flat spiral. Threaded objects are achieved by rotating around an internal axis, while helices and spirals are created by revolving around an external axis.

Presets:
-------

Behaviour:
Controls the coil using a combination of pitch, height and revolution. Also allows you to create a spiral rather than a helix. You can also modify the number of rotations.

\subsubsection{Emboss}
Engraves or embosses text or other geometry onto a face. You can raise the geometry out of the face, cut the geometry away from the face or do a combination. You can also control the appearance of the emboss.

\subsubsection{Decal}
Wraps an image to a face. By selecting multiple faces or Automatic Face Chain, you can also wrap the image over multiple faces.



\subsection{Primitives}
Primitives provide shortcuts to easily create basic features: a box, sphere, cylinder or torus. These features just reuse the Create features detailed above.