\section{3D Modelling}

This is where all the magic happens! Transform sketches into 3D features, surfaces and more.

\subsection{Creating Features}
In order to create a feature, you need to have valid sketches. The following features can be created:

\subsubsection{Extrude}
Takes a 2D cross-section and extrudes it into a prism. 

Surface Mode:
Instead of creating a prism, you can set the extrusion to create a surface instead, by selecting the relevant button on the top right of the popup dialog.

Profiles:
These are the cross-sections which are to be extruded. Select them by clicking on any desired closed areas within a single sketch.

From:
Choose the plane that the extrusion will start at (it does not have to be the sketch plane, although it will default to this).

Direction:
The extrusion can be set as in either direction, symmetric in both directions or asymmetric in the two directions.

Distance:
The distance of extrusion can be set as:

\begin{itemize}
    \item Manual Distance: specify the numerical distance of extrusion (for asymmetric extrusions you will need to specify a distance for each direction). You can click the arrow beside the input box to quickly measure dimensions, use an existing dimension for reference or use recently used values.
    \item Through All: the extrusion will pass through all objects (also can be used for cut or intersect extrusions).
    \item To: select the plane or feature that the extrusion will end at. If the feature is has multiple possible termination points (such as the closest and furthest side of a cylinder), select Alternate Solution to end the extrusion at the closest face of the feature
    \item To Next: the extrusion will end at the next feature or plane. Said feature or plane should entirely overlay the cross-section profile.
\end{itemize}

Output:
Determines the output of the extrusion, either:
\begin{itemize}
    \item Join: adds extrusion to the selected solid
    \item Cut: remvoes material from the selected solid
    \item Intersect: keeps the intersection between the extrusion and the selected solid
    \item New Solid: creates the extrusion as a new solid
\end{itemize}

Taper:
The extrusion tapers slightly inwards (negative values) or outwards (positive values).

Settings:
Additional settings can be accessed by the hamburger menu on the top right of the popup dialog. Of most note is the option to keep the sketch visible after extrusion - otherwise, the sketch will be consumed in the process.

\subsubsection{Revolve}
Revolves a section around a specified axis.

Surface Mode:
Instead of creating a solid feature, you can set the revolution to create a surface instead, by selecting the relevant button on the top right of the popup dialog.

Profiles:
These are the cross-sections which are to be revolved. Select them by clicking on any desired closed areas within a single sketch.

From:
Choose the plane that the revolution will start at (it does not have to be the sketch plane, although it will default to this).

Direction:
The revolution can be set as in either direction, symmetric in both directions or asymmetric in the two directions.

Distance:
The distance of revolution can be set as:

\begin{itemize}
    \item Manual Angle: specify the numerical angle of revolution (for asymmetric extrusions you will need to specify a distance for each direction). You can click the arrow beside the input box to quickly measure dimensions, use an existing dimension for reference or use recently used values.
    \item Full: a 360 degree revolution
    \item To: select the plane or feature that the extrusion will end at. If the feature is has multiple possible termination points (such as the closest and furthest side of a cylinder), select Minimum Solution to end the revolution at the closest face of the feature.
    \item To Next: the extrusion will end at the next feature or plane. Said feature or plane should entirely overlay the cross-section profile.
\end{itemize}

Output:
Determines the output of the revolution, either:
\begin{itemize}
    \item Join: adds revolution to the selected solid
    \item Cut: remvoes material from the selected solid
    \item Intersect: keeps the intersection between the revolution and the selected solid
    \item New Solid: creates the revolution as a new solid
\end{itemize}

\subsection{Sweep}
Takes a 2D cross section and sweeps it across a path to create a pipe-like feature.

Surface Mode:
Instead of creating a solid object, you can set the sweep to create a surface instead, by selecting the relevant button on the top right of the popup dialog.

Profiles:
These are the cross-sections which are to be sweep. Select them by clicking on any desired closed areas within a single sketch. They do not necessarily have to be connected but errors may occur if you try to sweep dispersed profiles along tricky paths. The

Path:
Choose the path (a 2D/3D curve or edge) for the profiles to follow. It does not necessarily have to intersect the profiles, but the path must intersect the profile plane.

Behaviour:
Determines how the profile follows the path:

\begin{itemize}
    \item Follow Path: the profile is swept along the path, where every cross section taken normal to the path stays constant. Adjust the twist and taper as necessary - 360 degree twist corresponds to one single twist, and positive/negative taper corresponds to a widening/narrowing of the sweep away from the start point.
    \item Fixed: the profile is swept along the path, where the cross section taken parallel to profile plane stays constant. Twist and taper are not available.
    \item Guide: 
\end{itemize}


Output:
Determines the output of the sweep, either:
\begin{itemize}
    \item Join: adds sweep to the selected solid
    \item Cut: remvoes material from the selected solid
    \item Intersect: keeps the intersection between the sweep and the selected solid
    \item New Solid: creates the sweep as a new solid
\end{itemize}



\subsection{Primitives}
Primitives provide shortcuts to easily create basic features: a box, sphere, cylinder or torus. These features just reuse the Create features detailed above.